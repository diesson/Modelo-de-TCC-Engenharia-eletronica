\documentclass[
	% -- opções da classe memoir --
	12pt,				% tamanho da fonte
	openright,			% capítulos começam em pág ímpar (insere página vazia caso preciso)\
	oneside,			% para impressão em verso e anverso
	a4paper,			% tamanho do papel. 
	% -- opções da classe abntex2 --
	chapter=TITLE,		% títulos de capítulos convertidos em letras maiúsculas
	% -- opções do pacote babel --
	english,			% idioma adicional para hifenização
	brazil				% o último idioma é o principal do documento
]{ifsc-tcc-abntex2}

%---------------------------------------------------------------------%
%---------------------------------------------------------------------%
% Informações de dados para CAPA e FOLHA DE ROSTO
%---------------------------------------------------------------------%
%---------------------------------------------------------------------%

\titulo{Título da Monografia}
\autor{Nome do Autor}
\local{Florianópolis}
\data{2021}
\orientador[Orientador:\\]{}
\coorientador[Coorientador:\\]{}
\tipotrabalho{Monografia (Graduação)}

% O preambulo deve conter o tipo do trabalho, o objetivo, o nome da instituição e a área de concentração 
\preambulo{Trabalho de conclusão de curso submetido ao Instituto Federal de Educação, Ciência e Tecnologia de Santa Catarina como parte dos requisitos para obtenção do título de engenheiro eletrônico}

\textoaprovacao{Este Trabalho foi julgado adequado para obtenção do Título de Engenheiro Eletrônico em março de 2021 e aprovado na sua forma final pela banca examinadora do Curso de Engenharia Eletrônica do instituto Federal de Educação Ciência, e Tecnologia de Santa Catarina.}


%---------------------------------------------------------------------%
% Início do documento
%---------------------------------------------------------------------%

\begin{document}

\selectlanguage{brazil}
\frenchspacing


% ----------------------------------------------------------
% ELEMENTOS PRÉ-TEXTUAIS
% ----------------------------------------------------------
\pretextual

\imprimircapa
\imprimirfolhaderosto* %(o * indica que haverá a ficha bibliográfica)

%---------------------------------------------------------------------%
% ATENÇÃO - Pergunte para a Biblioteca do IFSC
% Inserir a ficha bibliografica - 
%
% Para gerar a ficha catalográfica acesse:
% http://ficha.florianopolis.ifsc.edu.br/
% Precisa ser feito pelo navegador Mozilla Firefox
%---------------------------------------------------------------------%

%\imprimirficha{pdf/fichacatalografica.pdf}
\cleardoublepage

%---------------------------------------------------------------------%
% Inserir folha de aprovação
%---------------------------------------------------------------------%

\imprimiraprovacao
\cleardoublepage

%---------------------------------------------------------------------%
% Dedicatória
%---------------------------------------------------------------------%
\imprimirdedicatoria{Este trabalho é dedicado às crianças adultas que,\\quando pequenas, sonharam em se tornar cientistas.}
% ---

%---------------------------------------------------------------------%
% Agradecimentos
%---------------------------------------------------------------------%
\begin{agradecimentos}
	
	< Texto de Agradecimentos >
	
	%\marcador{add}{Tanta gente...}
\end{agradecimentos}
% ---

%---------------------------------------------------------------------%
% Epígrafe
%---------------------------------------------------------------------%
\begin{epigrafe}
	\vspace*{\fill}
	\begin{flushright}
		\textit{``Eu não falhei, encontrei 10 mil\\
			soluções que não davam certo.'' \\
			(Thomas A. Edison)}
	\end{flushright}
\end{epigrafe}


%---------------------------------------------------------------------%
% RESUMOS
%---------------------------------------------------------------------%
\setlength{\absparsep}{18pt} % ajusta o espaçamento dos parágrafos do resumo
\begin{resumo}
	
	No resumo deve-se ressaltar de forma clara e sintética a natureza e o objetivo do trabalho, o método que foi empregado, os resultados e as conclusões mais importantes, seu valor e originalidade. O resumo é a “apresentação concisa dos pontos relevantes de um texto. Constitui elemento essencial em textos de natureza técnico-científica” (ASSOCIAÇÃO BRASILEIRA DE NORMAS TÉCNICAS, 2003, p.3). O resumo não pode	ultrapassar 250 palavras. Abaixo do resumo devem aparecer as palavras-chave (mínimo três, máximo cinco, separadas por ponto final e iniciadas com letra maiúscula).
	
	\textbf{Palavras-chave}: mínimo três. máximo cinco. separadas por ponto final e iniciadas com letra maiúscula.
\end{resumo}
\begin{resumo}[Abstract]
	\begin{otherlanguage*}{english}
		This papper presents the study of electromagnetic compatibility techniques to reduce conducted and irradiated noise in a Buck Interleaved static converter. The proposed converter project had started in the Power Electronics discipline and has an output current of \SI{5}{\ampere}, an output voltage of \SI{5}{\volt} and a 25 watts of nominal power. In this work, the conducted and radiated emission tests of the designed converter were carried out, as well as the tests after the application of the proposed techniques, applied in isolation. In order to be possible to compare the results of the tests, a standardization of the tests was established, related to the position of the board and the cable. Among the proposed tests, some stood out for presenting significant results. A great solution presented was the reduction of the switching frequency with the use of a three-state switching cell converter. The techniques tested in isolation didn't show significant improvements simultaneously in the conducted and irradiated emission, however, when used in combination, they proved to be of great efficiency.
		
		\vspace{\onelineskip}
		
		\noindent 
		\textbf{Keywords}: Eletromagnetic compatibility. Power eletronics. Buck Interleaved. Three-state switching cell Buck Converter.
	\end{otherlanguage*}
\end{resumo}


%---------------------------------------------------------------------%
% inserir lista de ilustrações
%
% Nota: Somente deve aparecer em trabalhos com número de ilustrações  (desenhos,  esquemas,  fluxogramas,  fotografias,  gráficos,  mapas, plantas ou quadros) igual ou superior a cinco. Quando esse número for inferior a cinco, a lista de ilustrações é opcional. 
%---------------------------------------------------------------------%
\pdfbookmark[0]{\listfigurename}{lof}
\listoffigures*
\cleardoublepage

%---------------------------------------------------------------------%
% inserir lista de tabelas
%
% Nota:  Somente deve aparecer em trabalhos com cinco ou mais tabelas (quando inferior a cinco é opcional). 
%---------------------------------------------------------------------%
\pdfbookmark[0]{\listtablename}{lot}
\listoftables*
\cleardoublepage

%---------------------------------------------------------------------%
% inserir lista de códigos fonte (listings)
%---------------------------------------------------------------------%
\pdfbookmark[0]{\lstlistlistingname}{lol}
\listoflistings
\cleardoublepage

%---------------------------------------------------------------------%
% inserir lista de abreviaturas e simbolos
%
% Nota: Consiste  na  relação  alfabética  das abreviaturas e siglas utilizadas no texto, seguidas das palavras ou expressões correspondentes grafadas por extenso. Somente deve aparecer no trabalho se tiver um número de siglas igual ou superior a cinco (quando inferior a cinco é opcional).
% Obs: Se não houver nenhuma abreviatura ou símbolos sendo utilizados, haverá um erro ao tentar executar os dois comandos abaixo.
%---------------------------------------------------------------------%
\imprimirlistadeabreviaturas
\imprimirlistadesimbolos
\cleardoublepage
%\usepackage{Simbolos}

%---------------------------------------------------------------------%
% inserir o sumario
%---------------------------------------------------------------------%
\pdfbookmark[0]{\contentsname}{toc}
\tableofcontents*
\cleardoublepage

% ----------------------------------------------------------
% ELEMENTOS TEXTUAIS
% ----------------------------------------------------------
\textual

% Cria a lista de Símbolos/Unidades, Abreviaturas/Siglas, visível somente com \imprimirlistadeabreviaturas e \imprimirlistadesimbolos
\include{tex/00-Simbolos-Unidades}
\begin{siglas}
	\item[SMPS] \textit{Switched Mode Power Supply} - Fonte chaveada
	\item[CC] Corrente Contínua
	\item[CA] Corrente Alternada
	\item[EMC] \textit{Electromagnetic Compatibility} - Compatibilidade Eletromagnética
	\item[EMI] \textit{Electromagnetic Interference} - Interferência Eletromagnética
	\item[EMS] \textit{Electromagnetic Suscptibility} - Susceptibilidade Eletromagnética
	\item[CISPR] \textit{Comité International Special des Perturbations Radioélectriques} - Comitê Especial Internacional de Rádio Interferência
	\item[ESD] \textit{Electrostatic Discharge} - Descarga Eletrostática
	
	\item[PCI] Placa de Circuito Impresso
	
	\item[EMP] \textit{Electromagnetic Pulse} - Pulso Eletromagnético
	\item[FCC] \textit{Federal Communication Commission}
	
	\item[IEC] \textit{International Electrotechnical Commission}
	\item[LISN] \textit{Line Impedance Stabilization Network}
	\item[NFP] \textit{Near-Field Probe} - Sonda de Campo Próximo
	\item[DUT] \textit{Device Under Test} - Dispositivo Sob Teste
	\item[OATS] \textit{Open-Area Test Site} - Local de Teste de Área Aberta
	\item[SAC] \textit{Semi Anechoic Chamber} - Câmara Semi Anecoíca
	\item[CI] Circuito Integrado
	\item[IFSC] Instituto Federal de Educação Ciência e Tecnologia de Santa Catarina
	\item[VDC] \textit{Voltage Direct Current} - Tensão Contínua
	\item[VAC] \textit{Voltage Alternating Current} - Tensão Alternada
	\item[ABNT] Associação Brasileira de Normas Técnicas
	\item[abnTex] Normas para \LaTeX
	\item[PTH] Through-Hole Technology
	\item[SMD] Surface-Mount Device
	\item[ESR] Equivalent Series Resistance
\end{siglas}

%%%%%%%%%%%%%% Como usar o pacote acronym
% \ac{acronimo} -- Na primeira vez que for citado o acronimo, o nome completo irá aparecer seguido do acronimo entre parênteses. Na proxima vez somente o acronimo irá aparecer. Se usou a opção footnote no pacote, entao o nome por extenso irá aparecer aparecer no rodapé
%
% \acf{acronimo} -- Para aparecer com nome completo + acronimo
% \acs{acronimo} -- Para aparecer somente o acronimo
% \acl{acronimo} -- Nome por extenso somente, sem o acronimo
% \acp{acronimo} -- igual o \ac mas deixando no plural com S (ingles)
% \acfp{acronimo}--
% \acsp{acronimo}--
% \aclp{acronimo}--
%\begin{acronym}
%	\acro{ABNT}{Associação Brasileira de Normas Técnicas}
%	\acro{abnTeX}{ABsurdas Normas para TeX}
%	\acro{AC}{Autoridade Certificadora}
%	\acro{AES}{\textit{Advanced Encryption Standard}}
%	\acro{TLS}{\textit{Transport Layer Security}}
%	\acro{TPC}{Terceira Parte Confiável}
%\end{acronym}

% ----------------------------------------------------------
% Inclusão dos capítulos que estão em outros arquivos .tex
% ----------------------------------------------------------
%%%%%%%%%%%%%%%%%%%%%%%%%%%%%%%%%%%%%%%%%%%%%%%%%%%%%%%%%%%%%%%%%%%
%%%%%%%%%%%%%%%%%%%%%%%%%%%%%%%%%%%%%%%%%%%%%%%%%%%%%%%%%%%%%%%%%%%
\chapter{Introdução}
%%%%%%%%%%%%%%%%%%%%%%%%%%%%%%%%%%%%%%%%%%%%%%%%%%%%%%%%%%%%%%%%%%%
%%%%%%%%%%%%%%%%%%%%%%%%%%%%%%%%%%%%%%%%%%%%%%%%%%%%%%%%%%%%%%%%%%%
    
    A introdução abre o trabalho propriamente dito. Tem a finalidade de apresentar os motivos que levaram o autor a realizar a pesquisa, o problema abordado, os objetivos e a justificativa. O objetivo principal da introdução é situar o leitor no contexto da pesquisa. O leitor deverá perceber claramente o que foi analisado, como e por que, as limitações encontradas, o alcance da investigação e suas bases teóricas gerais. Ela tem, acima de tudo, um caráter didático de apresentar o que foi investigado, levando-se em conta o leitor a que se destina e a finalidade do trabalho. Assim, na introdução contextualize o tema, delimite o assunto, apresente um rápido histórico do problema e das soluções porventura já apresentadas, com breve revisão crítica das investigações anteriores; faça referência às fontes de material, aos métodos seguidos, às teorias ou aos conceitos que embasam o desenvolvimento e a argumentação, às eventuais faltas de informação, ao instrumental utilizado.
    A introdução deverá conter, ainda:
    
    \begin{alineas}
    	
    	\item Justificativa;
    	
    	\item Definição do problema;
    	    	
    	\item Objetivos: Neste item deverá ser indicado claramente o que se deseja fazer, o que se pretende alcançar. É fundamental que estes objetivos sejam possíveis de 29 serem atingidos. Geralmente se formula um objetivo geral articulando-o a outros objetivos mais específicos. Assim, pode-se dividi-los em:
    	
	    	\subitem Objetivo geral;
	    	
	    	\subitem Objetivos específicos: 
	    	
    \end{alineas}
    
    %%%%%%%%%%%%%%%%%%%%%%%%%%%%%%%%%%%%%%%%%%%%%%%%%%%%%%%%%%%%%%%%%%%
    \section{Justificativa}
    %%%%%%%%%%%%%%%%%%%%%%%%%%%%%%%%%%%%%%%%%%%%%%%%%%%%%%%%%%%%%%%%%%%
        
        trata-se da relevância, o motivo pelo qual tal pesquisa deve ser realizada. Justifica-se aqui a escolha do tema, a delimitação feita e a relação que o pesquisador possui com ele. Procura-se demonstrar a legitimidade, a pertinência, o interesse e a capacidade do pesquisador em lidar com o referido tema. Deve-se fazer o mesmo em relação ao problema e à hipótese, mostrando a relevância científica do tema para o pesquisador. Deve-se fazer, então, nesta parte, a justificativa para o tema, para o problema e para a hipótese, nos termos em que foram formulados na fase de elaboração do projeto de pesquisa.
        
    %%%%%%%%%%%%%%%%%%%%%%%%%%%%%%%%%%%%%%%%%%%%%%%%%%%%%%%%%%%%%%%%%%%
    \section{Descrição do problema}
    %%%%%%%%%%%%%%%%%%%%%%%%%%%%%%%%%%%%%%%%%%%%%%%%%%%%%%%%%%%%%%%%%%%
        
        Um problema decorre de um aprofundamento do tema. Ele deve delimitar a pesquisa. Diversos autores sugerem que o problema deve ter algumas características, tais como: a) deve ser formulado como pergunta – isso facilita sua identificação por quem consulta o projeto de pesquisa; b) deve ser claro e preciso; c) deve ser delimitado a uma dimensão variável, pois muitas vezes, o problema é formulado de uma maneira muito ampla, impossível de ser investigado (GIL, 2006).
        
    
    %%%%%%%%%%%%%%%%%%%%%%%%%%%%%%%%%%%%%%%%%%%%%%%%%%%%%%%%%%%%%%%%%%%
    \section{Objetivo geral}
    %%%%%%%%%%%%%%%%%%%%%%%%%%%%%%%%%%%%%%%%%%%%%%%%%%%%%%%%%%%%%%%%%%%

        
        procura-se determinar, com clareza e objetividade, o seu propósito com a realização da pesquisa. Deve-se estar atento ao fato de que nesta pesquisa, em nível de graduação ou pós-graduação, os propósitos são essencialmente acadêmicos, como mapear, identificar, levantar, diagnosticar, traçar o perfil ou historiar determinado assunto específico dentro de um tema. Um objetivo bem redigido explica o quê, com o quê (quem), por meio de quê, onde, quando sobre a pesquisa.
        
        Atenção! Inicie a frase com um verbo abrangente e no infinitivo, como: compreender, saber, avaliar, verificar, constatar, analisar,
        desenvolver, conhecer, entender, levantar, mapear, identificar.
    
    %%%%%%%%%%%%%%%%%%%%%%%%%%%%%%%%%%%%%%%%%%%%%%%%%%%%%%%%%%%%%%%%%%%
    \section{Objetivos específicos}
    %%%%%%%%%%%%%%%%%%%%%%%%%%%%%%%%%%%%%%%%%%%%%%%%%%%%%%%%%%%%%%%%%%%
    
    	significa aprofundar as intenções expressas no objetivo geral. Propõe-se mapear, identificar, levantar, diagnosticar, traçar o perfil ou historiar determinado assunto específico dentro de um tema. Assim, para elaborar os objetivos específicos deve-se:

    	\begin{alineas}
	    	\item detalhar o objetivo geral mostrando o que se pretende alcançar com a pesquisa;
	    	\item tornar operacional o objetivo geral, indicando exatamente o que será realizado na pesquisa;
	    	\item usar verbos que admitam poucas interpretações e no infinitivo, como: identificar, caracterizar, comparar, testar, aplicar, observar, medir, localizar, selecionar, distinguir. 
	    \end{alineas}
    
        Para esclarecimentos, verificar o Manual de Comunicação Científica do IFSC disponível em: 
        \url{https://intranet.ifsc.edu.br/images/file/manual_comunicacao_cientifica_IFSC_1_2016.pdf}.

É a parte principal do texto. Apresenta o assunto, fundamentação teórica, metodologia (materiais e métodos), os resultados e as respectivas discussões traçando relações com os trabalhos analisados na revisão de literatura.

%%%%%%%%%%%%%%%%%%%%%%%%%%%%%%%%%%%%%%%%%%%%%%%%%%%%%%%%%%%%%%%%%%%
%%%%%%%%%%%%%%%%%%%%%%%%%%%%%%%%%%%%%%%%%%%%%%%%%%%%%%%%%%%%%%%%%%%
\chapter{Fundamentação Teórica} \label{cap:fund}
%%%%%%%%%%%%%%%%%%%%%%%%%%%%%%%%%%%%%%%%%%%%%%%%%%%%%%%%%%%%%%%%%%%
%%%%%%%%%%%%%%%%%%%%%%%%%%%%%%%%%%%%%%%%%%%%%%%%%%%%%%%%%%%%%%%%%%%
    
    É uma análise comentada sobre o que já foi publicado sobre o assunto da pesquisa, buscando mostrar os pontos de vista convergentes e divergentes entre os autores. Traça-se um quadro teórico e elabora-se a estruturação conceitual que subsidiará o desenvolvimento  30 da pesquisa. A revisão de literatura permitirá um mapeamento de quem já escreveu e o que já foi escrito sobre o assunto ou o problema de pesquisa.

	Teste de Alguma abreviatura: \abreviatura*{CA}{corrente alternada}.
	
	\begin{figure}[H]
		\centering
		\caption{Um exemplo de figura}
		\includegraphics[width=\textwidth,height=240px,keepaspectratio]{pdf/noimage.png}
		\label{fig:esquematico_cbi}
		\indentedfont[15.2cm]{Elaboração própria (2021)}
	\end{figure}

	Alguns exemplos de equações:
	
	\[
		\binom{m+n}{m} = 
		\frac{(m+n)!}{m!n!} = 
		\frac{
			\overbrace{
				(m+n)(m+n-1)\cdots(n+1)
			}^{\clap{$m$ factors}}
		}{
			\underbrace{
				m(m-1)\cdots 1
			}_{\clap{$m$ factors}}}
	\]
	
	\begin{equation}
		\hat{x} = \hat{y} + a
	\end{equation}
	
	\begin{subequations}\label{eq-group1}
		\begin{align}
			\label{eq-lalala}
			\dot{x}
			&=	\begin{bmatrix}
				\dot{x_1} \\
				\dot{x_2} \\
				\dot{x_3} \\
			\end{bmatrix} 
			=	\begin{bmatrix}
				\dot{v_{C_1}} \\
				\dot{v_{C_2}} \\
				\dot{v_{C_3}} \\
			\end{bmatrix}
			=	\overbrace{
				\begin{bmatrix}
					0 & 559.44 & 0 \\
					-21.01 & -100.93 & 21.01 \\
					0 & 0 & -666.67
			\end{bmatrix}}^{A} x
			+	\overbrace{
				\begin{bmatrix}
					0 \\
					0 \\
					666.67
			\end{bmatrix}}^{B} u \\
			\label{eq-ssplanta_final}
			y
			&=	\overbrace{
				\begin{bmatrix}
					1 & 0 & 0 \\
			\end{bmatrix}}^{C} x
			+	\overbrace{
				\begin{bmatrix}
					0
			\end{bmatrix}}^{D} u
		\end{align}
	\end{subequations}
\chapter{Metodologia}
    
    É o caminho que se trilha, construindo, durante o percurso, os procedimentos e os instrumentos exigidos para se obter êxito no trabalho intelectual. É o momento da pesquisa em que se explicam, passo a passo, todos os procedimentos do estudo que permitiram que os resultados fossem atingidos, identificando os sujeitos com os quais foram coletados os dados, o modo como foram coletados, os instrumentos utilizados nessa coleta e a maneira como os dados foram analisados. 

    
\chapter{Métodos aplicados}

	Mostra-se como serão executados a pesquisa e o desenho metodológico que se pretende adotar: será do tipo quantitativa, qualitativa, descritiva, explicativa ou exploratória. Será um levantamento, um estudo de caso, uma pesquisa experimental, por exemplo. Segundo Gil 2006), a organização da metodologia depende do tipo de pesquisa a ser realizada. No entanto, alguns elementos devem ser apresentados, como:
	
	\begin{alineas}
	\item tipo de pesquisa: se é de natureza exploratória, descritiva ou explicativa.
	Convém, ainda, esclarecer acerca do tipo de delineamento a ser adotado pesquisa experimental, levantamento, estudo de caso, pesquisa bibliográfica);
	
	\item  população e amostra: envolve informações acerca do universo a ser estudado, da extensão da amostra e da maneira como será selecionada;
	\item coleta de dados: envolve a descrição das técnicas a serem utilizadas para a coleta de dados. Modelos de questionários ou testes deverão ser incluídos nessa parte. Se a pesquisa envolver técnicas de entrevista ou de observação, é também o momento de expor o assunto;
	\item análise dos dados: descrevem-se os procedimentos a serem adotados tanto
	para a análise quantitativa quanto qualitativa.
	\end{alineas}
\chapter{Resultados}
	Faz-se uma apresentação dos resultados a que se chegou a partir da pesquisa.
\chapter{Análise e Discussão}

	Faz-se uma exposição da análise obtida nos resultados da pesquisa, bem como uma discussão crítica a respeito deles.
\chapter{Considerações Finais}
    
    Nessa parte apresenta-se a síntese interpretativa dos principais argumentos usados, mostrando se os objetivos foram atingidos e se a(s) 31 hipótese(s) foi(foram) confirmada(s) ou rejeitada(s). Também se podem incluir recomendações e/ou sugestões para trabalhos futuros. Deve-se fazer uma rápida retomada dos capítulos que compõem o trabalho e uma espécie de autocrítica, fazendo um balanço a respeito dos resultados pela pesquisa.
    
    Atenção! A conclusão não constitui uma ideia nova ou um simples anexo sem importância ao trabalho. Pelo contrário, é nesse momento em que todas as ações do estudo são expostas, analisadas e finalizadas.
    
    Para melhor orientar-se, responda às seguintes questões:
    
    \begin{alineas}
	    \item a pesquisa resolve o problema, amplia a compreensão, mostra novas relações ou mesmo descobre outros problemas em relação ao originalmente escolhido?
	    \item a hipótese, ao final, foi confirmada ou refutada pela pesquisa?
	    \item os objetivos geral e específicos previamente definidos foram alcançados?
	    \item a metodologia de trabalho escolhida foi suficiente para a consecução de seus propósitos? houve necessidade, ao longo da pesquisa, de adotar outras técnicas ou procedimentos para lidar com situações não previstas?
	    \item a bibliografia previamente selecionada correspondeu às suas expectativas?
	\end{alineas}
    

% ----------------------------------------------------------
% ELEMENTOS PÓS-TEXTUAIS
% ----------------------------------------------------------
\postextual

% ----------------------------------------------------------
% Referências bibliográficas
% ----------------------------------------------------------
\bibliography{referencias}

% ----------------------------------------------------------
% Apêndices
% ----------------------------------------------------------
\begin{apendicesenv}
% Imprime uma página indicando o início dos apêndices
\partapendices

	\chapter{UM APÊNDICE}
	\label{ap:umalabel}
	
		Elemento opcional. "O(s) apêndice(s) são identificados por letras
		maiúsculas consecutivas, travessão e pelos respectivos títulos " (ASSOCIAÇÃO BRASILEIRA DE NORMAS TÉCNICAS, 2011). Os apêndices são textos e/ ou documentos elaborados pelo próprio autor para complementar o texto principal. Nos apêndices podem ser incluídos, por exemplo, questionários, modelos de entrevistas estruturadas ou transcrições de entrevistas utilizados
		no andamento da pesquisa.
		
		% Para incluir código, pesquisar sobre o pacote listingutf8
		% lstinputlisting{meucodigo.py}
	
\end{apendicesenv}

% ----------------------------------------------------------
% Anexos
% ----------------------------------------------------------
\begin{anexosenv}
% Imprime uma página indicando o início dos anexos
\partanexos

	\chapter{Um anexo}
	
	Elemento opcional. O anexo é um “texto ou documento não elaborado pelo autor, que serve de fundamentação, comprovação e ilustração” ASSOCIAÇÃO BRASILEIRA DE NORMAS TÉCNICAS, 2011). Também deve ser identificado por letras maiúsculas consecutivas, travessão e pelos respectivos títulos. No corpo do trabalho deve aparecer a indicação do anexo,
	sempre em ordem alfabética. 

\end{anexosenv}

%---------------------------------------------------------------------
% INDICE REMISSIVO
%---------------------------------------------------------------------
\phantompart
\printindex


\end{document}
\chapter{Metodologia}
    
    Este trabalho busca apresentar e estudar os possíveis problemas de interferência eletromagnética em conversores estáticos e possíveis soluções para a mitigação dos mesmos. Dessa forma, essa pesquisa pode ser classificada como exploratória com abordagem qualitativa. 
    
    Segundo \citeonline[p.~41]{ref:MEP_livro_gil}, uma pesquisa exploratória ``têm como objetivo proporcionar maior familiaridade com o problema, com vistas a torná-lo mais explícito ou a constituir hipóteses''. Já a abordagem qualitativa, para \citeonline[p.~111]{ref:MEP_livro_malhotra}, consiste em uma ``metodologia de pesquisa não estruturada e exploratória baseada em pequenas amostras que proporciona percepções e compreensão do contexto do problema''.
    
    Neste trabalho foi utilizado o \textit{software} KiCad para o projeto do \textit{layout} da placa de circuito impresso, a linguagem de programação Python para o cálculo dos elementos de filtro do conversor e para a geração de todos os gráficos utilizando dados obtidos nos ensaios de EMC e planilhas no \textit{software} SMath para o dimensionamento dos elementos magnéticos.
    
    Desenvolvida a PCB, foram realizados testes elétricos com o objetivo de validar os requisitos de projeto. Nesta etapa, utilizou-se equipamentos de medição como multímetro, osciloscópio, sonda diferencial de alta tensão e sonda de corrente de alta frequência. 
    
    Após validações da parte elétrica do conversor, realizou-se os testes de compatibilidade eletromagnética de emissão conduzida e emissão irradiada. Para as medidas de emissão conduzida, utilizou-se a LISN R\&S ENV216 em conjunto com o analisador de espectro R\&S HMS-X. Nas medida de emissão irradiada, foi utilizado uma câmara GTEM EMCTEST GTEM-750 e o receptor de EMI R\&S ESL3. 
    
    Todas as técnicas de mitigação de EMI testadas foram feitas de forma isolada, para que seja possível analisar o impacto da mesma sem interferência das técnicas anteriormente testadas. Por fim, foi realizado um teste com as técnicas combinadas que obtiveram melhores resultado na redução da EMI. Em todas as medidas utilizou-se como referência a norma CISPR22 classe B com a finalidade de facilitar a visualização dos resultados, porém, não é o objetivo desse trabalho se adequar a ela. 

    
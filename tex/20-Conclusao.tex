\chapter{Considerações Finais}
    No desenvolvimento de um conversor estático para fins comerciais, alguns cuidados do ponto de vista de EMC devem ser tomados pelo projetista, uma vez que o seu produto deve passar por algumas normas para que tenha permissão para comercialização. 
    
    Há diversas técnicas para soluções de problemas de EMC disponíveis para os diversos tipos de aplicações, tanto para a emissão conduzida quanto para a emissão irradiada. Dentre elas, algumas foram avaliadas neste trabalho, de forma isolada, em um conversor estático CC-CC do tipo Buck \interleaved. Os resultados obtidos mostraram-se interessantes, com nenhuma das técnicas mostrando-se uma solução perfeita para os problemas presentes, solucionando em determinadas faixas de frequências e sendo irrelevantes ou menos eficazes em outras. Porém, ao utilizar uma combinação das mesmas, resultados ainda mais interessantes podem ser obtidos, com melhoras tanto da emissão conduzida quanto da emissão irradiada. 
    
    Dentre as técnicas propostas, destacaram-se pela redução do ruído irradiado o uso de capacitores para a conexão dos dissipadores a referência do circuito e o uso de um núcleo de ferrite nos terminais de dreno dos transistores. Já no ensaio de emissão conduzida, a redução da frequência de chaveamento através do conversor 3SSC foi de grande destaque. Durante todos os testes, realizou-se medidas elétricas no circuito, avaliando o impacto das técnicas no seu funcionamento. Nesses testes verificou-se que não houve alterações no funcionamento, mantendo o conversor dentro dos requisitos estabelecidos.
    
    Durante o trabalho, o uso da padronização da posição dos cabos e da placa foi de grande valia, uma vez que a sua variação durante a medição de emissão irradiada ocasionava em uma grande mudança nos valores medidos. Para o projeto dos indutores, a variação do tamanho do entreferro foi de grande influência na emissão irradiada, e é um fator que muitas vezes não recebe a devida atenção pelos projetistas. 
    
    Apesar das diversas técnicas de EMC apresentadas nesse projeto, muitas outras ainda podem ser estudadas no conversor proposto ou variações das mesmas. Uma delas é a conexão dos dissipadores a referência sem o uso de um capacitor, desde que seja feito o seu isolamento, e o uso de capacitores de diferentes valores de capacitância. Dessa forma, pode-se avaliar melhor o impacto da capacitância escolhida na redução das emissões. 
    
    Também, mostra-se relevante para estudos futuros um melhor estudo do impacto do entreferro dos indutores na emissão irradiada, podendo ser estudado tanto a variação do entreferro quanto a mudança dos tipos de núcleo. Ainda, sugere-se um maior estudo do conversor 3SSC, com uma análise em diferentes frequências, tanto do ponto de vista da compatibilidade eletromagnética quanto do ponto de vista de potência.
    
    Assim, com todos os objetivos propostos alcançados, este trabalho buscou agregar conhecimento para os próximos trabalhos na área, tanto para os estudos de conversores Buck \interleaved e conversores com 3SSC quanto no estudo da EMI.
    
    
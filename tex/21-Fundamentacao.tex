%%%%%%%%%%%%%%%%%%%%%%%%%%%%%%%%%%%%%%%%%%%%%%%%%%%%%%%%%%%%%%%%%%%
%%%%%%%%%%%%%%%%%%%%%%%%%%%%%%%%%%%%%%%%%%%%%%%%%%%%%%%%%%%%%%%%%%%
\chapter{Fundamentação Teórica} \label{cap:fund}
%%%%%%%%%%%%%%%%%%%%%%%%%%%%%%%%%%%%%%%%%%%%%%%%%%%%%%%%%%%%%%%%%%%
%%%%%%%%%%%%%%%%%%%%%%%%%%%%%%%%%%%%%%%%%%%%%%%%%%%%%%%%%%%%%%%%%%%
    
    É uma análise comentada sobre o que já foi publicado sobre o assunto da pesquisa, buscando mostrar os pontos de vista convergentes e divergentes entre os autores. Traça-se um quadro teórico e elabora-se a estruturação conceitual que subsidiará o desenvolvimento  30 da pesquisa. A revisão de literatura permitirá um mapeamento de quem já escreveu e o que já foi escrito sobre o assunto ou o problema de pesquisa.

	Teste de Alguma abreviatura: \abreviatura*{CA}{corrente alternada}.
	
	\begin{figure}[H]
		\centering
		\caption{Um exemplo de figura}
		\includegraphics[width=\textwidth,height=240px,keepaspectratio]{pdf/noimage.png}
		\label{fig:esquematico_cbi}
		\indentedfont[15.2cm]{Elaboração própria (2021)}
	\end{figure}

	Um exemplo de equação matricial:
	
	\[
	g^{(e)}(x) = \overbrace{\bigl( 1 + \tfrac{x^2}{2!} + \tfrac{x^4}{4!} + \dots \bigr)}^{\text{red}}
	\overbrace{\bigl( 1 + \tfrac{x}{1} + \tfrac{x^2}{2!} + \dots \bigr)}^{\text{green}}
	\overbrace{\bigl( 1 + \tfrac{x}{1} + \tfrac{x^2}{2!} + \dots \bigr)}^{\text{white}}
	\]
	
	$$
	\left(
	\begin{array}{c}
		m+n\\
		m
	\end{array}
	\right)
	= \frac{(m+n)!}{m!n!}
	= \frac{\overbrace{(m+n)(m+n-1)\cdots(n+1)}^{\mbox{$m$ factors}}}{\underbrace{m(m-1)\cdots 1}_{\mbox{$m$ factors}}}
	$$

	\begin{eqnarray}
		\nonumber
		\dot{x} = 
		\begin{bmatrix}
			\dot{x_1} \\
			\dot{x_2} \\
			\dot{x_3} \\
		\end{bmatrix} = 
		\begin{bmatrix}
			\dot{v_{C_1}} \\
			\dot{v_{C_2}} \\
			\dot{v_{C_3}} \\
		\end{bmatrix} =
		\overbrace{
			\begin{bmatrix}
				0 & 559.44 & 0 \\
				-21.01 & -100.93 & 21.01 \\
				0 & 0 & -666.67
		\end{bmatrix}}^{A} x
		+
		\overbrace{
			\begin{bmatrix}
				0 \\
				0 \\
				666.67
		\end{bmatrix}}^{B} u \\
		\label{eq-ssplanta_final}
		y = 
		\overbrace{
			\begin{bmatrix}
				1 & 0 & 0 \\
		\end{bmatrix}}^{C} x
		+
		\overbrace{
			\begin{bmatrix}
				0
		\end{bmatrix}}^{D} u
	\end{eqnarray}